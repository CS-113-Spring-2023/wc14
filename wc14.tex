\documentclass[a4paper,addpoints]{exam}

\usepackage{amsmath,amssymb,amsthm}
\usepackage{framed}
\usepackage{geometry}
\usepackage{titling}

\theoremstyle{definition}
\newtheorem{definition}{Definition}
\theoremstyle{claim}
\newtheorem{claim}{Claim}

\runningheader{CS/MATH 113}{WC14: Structural Induction}{\theauthor}
\runningheadrule
\runningfootrule
\runningfooter{}{Page \thepage\ of \numpages}{}

\printanswers

\title{Weekly Challenge 14: Structural Induction\\CS/MATH 113 Discrete Mathematics}
\author{team-name}  % <== for grading, replace with your team name, e.g. q1-team-420
\date{Habib University | Spring 2023}

\qformat{{\large\bf \thequestion. \thequestiontitle}\hfill[\thepoints]}

\begin{document}
\maketitle

\begin{questions}

\titledquestion{$k$-ary tree}[10]
  Definition 5 in Section 5.3 of our textbook defines a \textit{full binary tree}. We extend this definition to a \textit{full $k$-ary tree} as follows.
  \begin{framed}
    \begin{definition}[Full $k$-ary tree]
      \begin{description}
      \item[Basis Step] There is a full $k$-ary tree consisting only of a single vertex $r$.
      \item[Recursive Step]  If $T_1,T_2, T_3,\ldots,T_k$ are disjoint full $k$-ary trees, there is a full $k$-ary tree, denoted by $T_1\cdot T_2\cdot T_3\cdot\ldots\cdot T_k$, consisting of a root $r$ together with edges connecting the root to each of the roots of $T_1,T_2, T_3,\ldots,T_k$.
      \end{description}
    \end{definition}
  \end{framed}
  We also introduce the following definitions of nodes in a tree.
  \begin{definition}[Leaf node]
    A leaf node in a tree is a node that no children.
  \end{definition}
  \begin{definition}[Internal node]
    An internal node in a tree is a node that is not a leaf node.
  \end{definition}

  Use structural induction to prove the following claim.
  \begin{claim}
    The number of internal nodes in a full $k$-ary tree with $n$ leaves is $\frac{n-1}{k-1}$.
  \end{claim}

  \begin{solution}
    % Enter your solution here.
  \end{solution}

\end{questions}


\end{document}

%%% Local Variables:
%%% mode: latex
%%% TeX-master: t
%%% End:
